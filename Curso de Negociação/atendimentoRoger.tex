\documentclass[12pt]{article}
\usepackage[brazilian]{babel}
\usepackage[utf8]{inputenc}
\usepackage[T1]{fontenc}

\begin{document}
\title{%
  Guia de Negociação - Insoft \\\hfill \break
  \large 1.Atendimento e tratamento}

\author{Roger Lenke}
\maketitle


No âmbito empresarial, principalmente na venda de produtos, existe uma concorrência extremamente ampla. Se consideramos a área da tecnologia, percebemos cada vez
mais que as empresas se aproximam de um patamar similar no aspecto de hardware. Isso abre um grande leque de opções quando o objetivo é adquirir um serviço ou resolver um problema.
Porém, se são oferecidos praticamente os mesmos serviços, com uma qualidade final do produto similar, o que irá atrair um cliente para determinada empresa? A personalização e o 
\emph {atendimento}.
\hfill \break

Antender, do latim, \emph{attendere}, significa prestar atenção. Esse significado é bom o suficiente para que um indivíduo consiga absorver a essência do atendimento: é dar atenção a um potencial comprador. A atenção não se resume, no entanto, a estar presente na tomada de decisão do cliente. A atenção é estar disposto a resolver o problema dentro das capacidades da empresa, compreender o que é desejado, \emph{tratar bem} e buscar a melhor solução possível. Com esses quatro aspectos, a chance da empresa receber um \emph{feedback} positivo é muito maior.
\hfill \break

Existem alguns aspectos que são essenciais para um bom atendimento. Esses aspectos serão relatados um por um, a seguir:
\hfill \break

\title{%
\large 1.1 Requisitos}
\hfill \break

A compreensão de quais são as reais necessidades do comprador é um aspecto aparentemente óbvio, mas de extrema importância. Inúmeras vezes, o cliente \emph{desconhece} sua real necessidade. Conhecendo quais são os desejos e objetivos do mesmo pode levar a uma solução mais eficiente do problema, mesmo que esta não seja oferecida pela empresa.
\hfill \break

Esses aspectos a serem analisados são costumeiramente chamados de \emph{requisitos}, no ramo da tecnologia. A análise de requisitos é um estudo das necessidades do comprador para se encontrar uma definição eficiente do produto desejado. Algumas perguntas que são úteis para determinar quais são as necessidades do cliente são: \hfill \break
- Qual é o produto desejado? \hfill \break
- Qual é a finalidade do produto? \hfill \break
- Qual objetivo o cliente deseja alcançar com a aquisição do mesmo?
\hfill \break

No ramo tecnológico, a análise de requisitos é mais complexa, consistindo em: \hfill \break

- Reconhecer o problema. \hfill \break
- Análise do problema. \hfill \break
- Modelagem ou prototipação do produto-solução. \hfill \break
- Especificação de requisitos. \hfill \break
- Revisão. \hfill \break

\title{%
\large 1.2 Comunicação}
\hfill \break

A comunicação com os compradores é outro aspecto que parece óbvio na essência do atendimento. Porém, é importante perceber que o cliente é um \emph{ser humano}, ou seja, é capaz de identificar qualquer característica negativa presente na comunicação em um atendimento. Por isso, a postura, a entonação de voz e a expressão corporal são aspectos tão importantes e amplamente estudados quando o foco é como se apresentar em relação a um público. \hfill \break

A vestimenta e a aparência também fazem parte de uma boa comunicação, como o uso de uniformes, dando a ideia de organização e criando uma imagem positiva. A \emph{empatia} é outro aspecto fundamental na comunicação para com os compradores, fazendo com que o cliente se sinta confortável para buscar uma solução para o seu problema, além de aumentar o grau de confiança que a empresa possui. A empatia também torna possível um outro aspecto fundamental do atendimento, a \emph{personalização}.\hfill \break

\title{%
\large 1.3 Personalização}
\hfill \break

A personalização do atendimento é provavelmente uma das características que mais causa retorno do cliente à empresa. É importante que todos os compradores se sintam individualmente especiais no momento em que estão sendo atendidos, para que possuam a sensação de um esforço verdadeiro na tentativa de solução do seu problema. Por isso, a empatia é tão importante.\hfill \break

Essa personalização se torna mais fácil aliada ao \emph{cadastro de clientes} que possibilita adquirir informações estratégias sobre as preferências de cada um dos mesmos.\hfill \break
\hfill \break
\hfill \break

Esses três aspectos são extremamente importantes para a realização de um atendimento eficiente. A depender da área de atuação, podem existir outros aspectos que também se tornam essenciais. Porém, é necessário ter a ideia concreta que o atendimento é a palavra chave para o crescimento, e ter uma imagem positiva para com os clientes é ter um grande leque de marketing livre de custos.

\end{document}
