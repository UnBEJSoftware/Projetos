\documentclass[12pt]{article}
\usepackage[brazilian]{babel}
\usepackage[utf8]{inputenc}
\usepackage[T1]{fontenc}

\begin{document}
\title{%
Scrum}


\author{Roger Lenke}
\maketitle

\title{%
\large Introdução}\hfill \break

O scrum é uma metodologia para o planejamento e gestão de projetos de software. É utilizado principalmente em trabalhos onde é impossível prever tudo o que irá acontecer no decorrer do desenvolvimento, ou quando existe um conhecimento não tão largo dos requisitos e/ou a tecnologia a ser utilizada é extremamente complexa.\hfill \break

Quando o scrum é utilizando, há uma grande transparência de quais processos estão em desenvolvimento, e todos os envolvidos conseguem ver com clareza qual é o status do determinado projeto, além da clareza em perceber os requisitos. Além disso, há uma constante análise do que está sendo produzido. Por fim, uma característica pertinente é que há uma adaptação do que está sendo desenvolvido durante a realização do projeto.\hfill \break

\title{%
\large Papéis}\hfill \break
O scrum é constituido de três papéis básicos, necessários para que o scrum \emph{exista}. São eles: \hfill \break

1. Scrum Master:\hfill \break

É um facilitador da comunicação do Product Owner com o Dev Team. \hfill \break

É importante perceber que o Scrum Master não é o gerente da equipe de desenvolvimento, mas sim um membro designado para remover qualquer obstáculo que esteja impedindo a equipe de alcaçar seus objetivos. O Scrum Master ajuda a equipe a se manter criativa, além de deixar claro que os objetivos alcançados sejam visíveis para o Product Owner.\hfill \break

2. Product Owner:\hfill \break

O Product Owner é o responsável por fazer decisões de mercado/negócios, além de definir quais são as prioridades do Dev Team, quais recursos serão construídos e qual a ordem dessa construção. Além disso, o Product Owner é o responsável pelos itens do \emph{product backlog}, que será explicado adiante.\hfill \break

3. Dev Team:\hfill \break

O Dev Team é responsável pela construção real do produto. É responsável também pela sua própria organização para que o trabalho seja concluído. A cada objetivo a ser completado, a equipe precisa decidir qual será a maneira utilizada para que o trabalho seja concluído, de forma autônoma. \hfill \break

\title{%
\large Dinâmica do scrum}\hfill \break

O product owner é responsável por fornecer a visão do produto, num \emph{macro planejamento} de forma a especificar com clareza quais são os objetivos a serem atingidos. Posteriormente, essa visão macro precisa ser desmembrada, com a ajuda do scrum master, em todas as funções que serão necessárias no produto. Essa lista de funções é chamada de \emph{product backlog}, na qual as funções são organizadas por \emph{prioridade}.\hfill \break

Um projeto utilizando o scrum é planejado em \emph{sprints}. Sprints são períodos de tempo onde alguns itens do product backlog serão selecionados, produzidos e entregues. Para realizar esse planejamento, são observados os \emph{eventos de duração fixa}, ou \emph{time-boxes}. É ideal que os sprints tenham uma duração fixa e igual.\hfill \break

Antes do início de cada sprint, existe uma reunião de planejamento do sprint, chamada de \emph{sprint planning}. Nessa reunião, é criado o \emph{sprint backlog}. ou seja, são definidas quantas e quais funcionalidades podem ser feitas naquele sprint.\hfill \break

As demais funcionalidades dos sprints seguintes são definidas seguindo a ordem de prioridade do product backlog. Se uma mudança for necessária, ela deverá ser adicionada no product backlog em uma prioridade definida. Esse processo segue até que todo o product backlog tenha sido concluído, e o produto final esteja pronto.\hfill \break

Outra característica do scrum é que todos os dias, existe uma reunião onde cada membro deve responder a três questões:\hfill \break
- O que eu fiz ontem?\hfill \break
- O que eu vou fazer hoje?\hfill \break
- Existe algum obstáculo?\hfill \break

Dessa maneira, cada membro pode reconhecer como está se dando o progresso do sprint.

No fim do sprint, existem duas atividades necessárias para a entrega da funcionalidade: o \emph{sprint review}, que é a verificação se o que está sendo feito está de acordo com o esperado, e a \emph{retrospectiva}, que é a avaliação do que foi feito e poderia ser melhorado para um sprint mais eficiente. \hfill \break

Após a conclusão de todo esse processo, o produto estará finalizado e os envolvidos terão um ótimo feedback para constante progresso no desenvolvimento de novos projetos.
\end{document}
